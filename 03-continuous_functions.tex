\documentclass{article}
\usepackage[utf8]{inputenc}
\usepackage[spanish]{babel}
\usepackage{amsthm}
\usepackage{amssymb}
\usepackage{amsmath}
\usepackage{fancyhdr}
\usepackage{graphicx}
\usepackage[dvipsnames, table]{xcolor}
\usepackage[framemethod=tikz]{mdframed}
\usepackage{multicol}
\usepackage{tabularx}
\usepackage{pifont}
\setlength{\tabcolsep}{3pt}
\decimalpoint
\newcommand{\xmark}{\ding{55}}
\definecolor{mycolor}{rgb}{0.122, 0.435, 0.698}

\newmdenv[innerlinewidth=0.5pt, roundcorner=4pt,linecolor=mycolor,innerleftmargin=6pt,
innerrightmargin=6pt,innertopmargin=6pt,innerbottommargin=6pt]{mybox}

\newcommand\restr[2]{{% we make the whole thing an ordinary symbol
		\left.\kern-\nulldelimiterspace % automatically resize the bar with \right
		#1 % the function
		\vphantom{\big|} % pretend it's a little taller at normal size
		\right|_{#2} % this is the delimiter
}}

\setlength\parindent{0pt}

\newtheorem{definition}{Definición}
\newtheorem{proposition}{Proposición}

\renewcommand{\headrulewidth}{0.4pt}

\begin{document}
\date{27 de marzo - 10 de abril de 2020}
\title{ \textbf{Topología} \\
Semanas 4 y 5: Continuidad y homeomorfismos}
\maketitle	

\begin{mybox}
	\textbf{1. } Sean $X$ y $Y$ espacios topológicos y supongamos que $\mathcal{B}$ es una base para la topología sobre $Y$. Una función $f: X \rightarrow Y$ es continua si y solo si $f^{-1}(B)$ es abierto para todo $B \in \mathcal{B}$.
\end{mybox}	
\begin{proof}
	$(\Rightarrow)$ Si $f$ es continua, se sigue que la preimagen de cualquier abierto en $Y$ es abierta en $X$. En particular, la preimagen de cualquier elemento de la base $\mathcal{B}$ es abierta.\\
	$(\Leftarrow)$ Ahora tomemos un abierto $V$ en $Y$ y mostremos que $f^{-1}(V)$ es abierto en $X$. Entonces,
	$$ V = \bigcup_{i \in J} B_{i} ,$$
	de forma que 
	$$ f^{-1}(V) = \bigcup_{i \in J} f^{-1}(B_{i}) .$$
	Al ser la unión de conjuntos abiertos, se sigue que $f^{-1}(V)$ es abierto. 
\end{proof}

\begin{mybox}
	\textbf{2. } Sea $f: \mathbb{R} \rightarrow \mathbb{R}$ una función continua tal que $f(q) = 0$ para todo racional $q$. Mostrar que $f(x) = 0$ para todo $x \in \mathbb{R}$.  	
\end{mybox}	

\begin{proof}
	Puesto que $f$ es continua, se cumple que 
	$$ f(\text{cl}(\mathbb{Q})) \subseteq \text{cl}(f(\mathbb{Q})), $$
	es decir,
	$$ f(\mathbb{R}) \subseteq \{ 0 \}. $$
	Por la hipótesis, sabemos que $\{0 \} \subseteq f(\mathbb{R})$. A partir de esto, deducimos que $f(\mathbb{R}) = \{ 0 \}$.
\end{proof}

\begin{proof}
	Sea $x$ irracional y supongamos que $f(x) \neq 0$. Puesto que $\mathbb{R}$ es $T_{1}$, podemos encontrar un entorno $U$ de $f(x)$ que no contenga a $0$. Se sigue, por la continuidad de $f$, que $f^{-1}(U)$ es abierto. Del hecho que $U$ es abierto, podemos deducir que existe un racional $q$ en $f^{-1}(U)$. Esto implicaría que $0 = f(q) \in U$, lo cual es una contradicción. 
\end{proof}


\begin{mybox}
	\textbf{3. }  Supongamos que $f: X \rightarrow Y$ es continua. Si $x$ es un punto límite del subconjunto $A$ de $X$, ¿es necesariamente cierto que $f(x)$ es un punto límite de $f(A)$? 
\end{mybox}	

$\bullet$ No es necesariamente cierto. Podría ser que el conjunto $Y$ no tenga puntos límite. Por ejemplo, $f: \mathbb{R} \rightarrow \mathbb{R}$ dada por $f(x) = 1$ es una función continua. Consideremos $A = (2,3)$.  Entonces, $2$ es un punto límite de $A$, pero $f(2) = 1$ no es un punto límite de $f(A) = \{ 1 \}$: este conjunto no tiene puntos límite. 

\begin{mybox}
	\textbf{4. } Consideremos $\text{id}: (X, \mathcal{T}_{1}) \rightarrow (X, \mathcal{T}_{2})$. Se cumplen las siguientes proposiciones: \\
	
	(i) La función $\text{id}$ es continua si y solo si $\mathcal{T}_{1}$ es más fina que $\mathcal{T}_{2}$. \\
	
	(ii) La función id es un homeomorfismo si y solo si $\mathcal{T}_{1} = \mathcal{T}_{2}$.  
\end{mybox}	
\begin{proof}
	(i) $(\Rightarrow)$ Supongamos que id es continua y consideremos un abierto $V$ en $\mathcal{T}_{2}$. Entonces, $\text{id}^{-1}(V) = V$ es abierto en $\mathcal{T}_{1}$, de modo que $\mathcal{T}_{1}$ debe ser más fina que $\mathcal{T}_{2}$. \\
	$(\Leftarrow)$ Ahora supongamos que $\mathcal{T}_{1}$ es más fina que $\mathcal{T}_{2}$. Entonces, al tomar un abierto $V$ de $\mathcal{T}_{2}$, sabemos que $\text{id}^{-1}(V) = V$ también será abierto en $\mathcal{T}_{1}$, de lo que concluimos que id es continua. \\
	
	(ii) $(\Rightarrow)$ Supongamos que id es un homeomorfismo. Entonces, si $U \in \mathcal{T}_{1}$, sabemos que $\text{id}(U) = U \in \mathcal{T}_{2}$, de forma que $\mathcal{T}_{1} \subseteq \mathcal{T}_{2}$.   De forma análoga, si $V \in \mathcal{T}_{2}$, $\text{id}^{-1}(V) = V$ es abierto en $\mathcal{T}_{1}$, de manera que $\mathcal{T}_{2} \subseteq \mathcal{T}_{1}$. \\
	$(\Leftarrow)$ Si $\mathcal{T}_{1} = \mathcal{T}_{2}$, se sigue inmediatamente $\text{id} = \text{id}^{-1}$ es continua.  
\end{proof}

\newpage

\begin{mybox}
	\textbf{5. } Todos los subespacios $(a, b)$  de $\mathbb{R}$ son homeomorfos.
\end{mybox}	
\begin{proof}
	$\bullet$ Sea $(a, b)$ arbitrario. Mostramos que $(0,1) \cong (a, b)$. Consideremos la función $f: (0, 1) \rightarrow (a, b)$ dada por $f(x) = a(1-x) + bx$. La función $g: (a, b) \rightarrow (0, 1)$ dada por $g(x) = (x- a)/ (b - a)$ es inversa de $f$, de lo que deducimos que $f$ es biyectiva. \\
	 Ahora mostramos que $f$ es continua. El conjunto 
	$$ \mathcal{B} = \{ (a, b) \cap (c, d) \subseteq \mathbb{R} \mid c < d \} = \{ (c, d) \subseteq \mathbb{R} \mid a \leq c < d \leq b \}$$
	es una base para la topología sobre $(a, b)$.  Consideremos $(c, d) \subseteq (a, b)$. Entonces, 
	$$ f^{-1}((c, d)) = \left( \frac{c - a}{b - a}, \frac{d -a}{b - a} \right) \subseteq (0, 1) $$
	es abierto en la topología sobre $(0, 1)$. Así, la preimagen de todo básico de $(a, b)$ es abierta, de lo que se sigue que $f$ es continua. \\
	Para mostrar que $g$ es continua, tomemos un intervalo $(k, l) \subseteq (0,1)$. Entonces,
	$$ g^{-1}((k, l )) = ( a(1- k) + bk, a(1- l) + bl ) \subseteq (a, b), $$
	de forma que $g^{-1}$ es continua. \\
	Así, hemos demostrado que $(0, 1) \cong (a, b)$. Puesto que la relación entre dos espacios de ser homeomorfos es una relación de equivalencia, podemos deducir que todos los intervalos abiertos de $\mathbb{R}$ son homeomorfos. 
\end{proof}

\begin{mybox}
	\textbf{6. } Sea $\{ A_{i} \}_{i \in J}$ una familia de subconjuntos de $X$ tal que $X = \bigcup_{i \in J} A_{i}$. Sea $f: X \rightarrow Y$ y supongamos que $\restr{f}{A_{i}}$ es continua para todo $i \in J$. \\
	
	(a) Mostrar que si la colección $\{ A_{i} \}_{i \in J}$ es finita y cada $A_{i}$ es cerrado, entonces $f$ es continua. \\
	
	(b) Encontrar un ejemplo donde la colección $\{ A_{i} \}_{i \in J}$ sea contable y cada $A_{i}$ cerrado, pero $f$ no sea continua.  \\
	
	(c) Una colección de subconjuntos de un espacio $X$ se denomina \textbf{localmente finita} si cada $x \in X$ tiene un entorno que interseca solo a una cantidad finita de conjuntos en la colección. Mostrar que si $\{ A_{i} \}_{i \in J}$ es localmente finita y cada $A_{i}$ es cerrado, entonces $f$ es continua.
	
\end{mybox}	

\begin{proof}
	Supongamos que $\{ A_{i} \}_{i \in J}$ es finita y que cada uno de estos conjuntos es cerrado. Sea $C$ un cerrado de $Y$. Entonces, 
	$$ f^{-1}(C) = \bigcup_{i \in J} ( \restr{f}{A_{i}} )^{-1}(C) .$$
	Al ser una unión finita de conjuntos cerrados, se sigue que $f^{-1}(C)$ es cerrado. Por tanto, la función $f$ es continua.
\end{proof}

$\bullet$ Consideremos $X = \mathbb{Z}$ con la topología cofinita  y $f: \mathbb{Z} \rightarrow \mathbb{R}$ dada por $f(m) = m$, donde $\mathbb{R}$ está equipado con la topología usual. Entonces, podemos escribir

$$ \mathbb{Z} = \bigcup_{m \in \mathbb{Z}} \{ m \} $$

Al ser conjuntos finitos, todos los conjuntos unipuntuales $\{m \}$ son cerrados en $\mathbb{Z}$ con la topología cofinita.

Notamos que cada restricción $\restr{f}{\{m \}}$ es continua: todas son funciones constantes. Sin embargo, 

$$ f^{-1}((1, 5)) = \{ 2, 3, 4 \}.$$

Notamos que $(1, 5)$ es abierto en $\mathbb{R}$ con la topología usual, pero $\{ 2, 3, 4 \}$ no es abierto en $\mathbb{Z}$ con la topología cofinita. \\

El ejemplo anterior está relacionado con la siguiente proposición. \\

\textbf{Proposición. } Supongamos que $f:X \rightarrow Y$ es continua, donde $X$ es un conjunto infinito equipado con la topología cofinita y $Y$ es un espacio de Hausdorff. Entonces, $f$ es constante. 
\begin{proof}
Procedemos por contradiccíón. Supongamos que existen $x_{1}$ y $x_{2}$ en $X$ tales que $f(x_{1}) \neq f(x_{2})$. Puesto que $Y$ es de Hausdorff, existen entornos disjuntos $U$ de $f(x_{1})$ y $V$ de $f(x_{2})$. Se sigue que el conjunto $f^{-1}(U) \cap f^{-1}(V)$ es abierto en $X$. Sin embargo, $f^{-1}(U) \cap f^{-1}(V) =f^{-1}(U \cap V) = f^{-1}(\varnothing) = \varnothing  $.  Entonces,
$$ X = X \backslash \varnothing = X \backslash (f^{-1}(U) \cap f^{-1}(V)) = X \backslash f^{-1}(U) \cup X \backslash f^{-1}(V)$$ 
lo que contradice el hecho que $X$ es infinito. 
\end{proof}

Ahora vamos a la demostración de (c).

\begin{proof}
	Supongamos que $\{ A_{i} \}_{i \in J}$ es localmente finita y que cada $A_{i}$ es cerrado. Sea $C$ cerrado en $Y$. Definimos $C_{i} = (\restr{f}{A_{i}})^{-1}(C)$ para cada $i \in I$. Por hipótesis, esta es una colección de conjuntos cerrados. Además, es localmente finita: consideremos $x$ en $X$ y el entorno $U$ de $x$ que interseca solo a una cantidad finita de elementos de $\{A_{i} \}_{i \in I}$. Es decir, el conjunto 
	$$ \{ i \in I \mid U \cap A_{i} \neq \varnothing \} $$
	es finito. Notamos que para cada $i \in I$, $C_{i} \subseteq A_{i}$. Consideramos el conjunto $U$ como el entorno candidato de $x$ que interseca solo a una cantidad finita de elementos de $\{ C_{i} \}_{i \in I}$. Entonces, se cumple que 
	$$\{ i \in I \mid U \cap C_{i} \neq \varnothing \} \subseteq  \{ i \in I \mid U \cap A_{i} \neq \varnothing \},$$
	de forma que el conjunto de la derecha debe ser finito y $\{ C_{i} \}_{i \in I}$ localmente finita. Así, 
	$$f^{-1}(C) = \bigcup_{i \in I} (\restr{f}{A_{i}})^{-1}(C)  = \bigcup_{i \in I} C_{i} $$ 
	es cerrado, al ser la unión de una colección localmente finita de cerrados. Por tanto, la función $f$ es continua.  
\end{proof}

\begin{mybox}
	\textbf{7.} Sea $X$ un espacio topológico y $\mathcal{A} \subseteq \mathcal{P}(X)$ localmente finita. Entonces, toda colección $\mathcal{B} \subseteq \mathcal{A}$ es localmente finita. 
\end{mybox}	
\begin{proof}
	Consideremos $x$ en $X$ y un entorno $U$ de $x$ que solo interseca una cantidad finita de elementos de $\mathcal{A}$. Es decir, el conjunto
	$$ \{ A \in \mathcal{A} \mid U \cap A \neq \varnothing \} $$
	es finito. Además, contiene al conjunto
	$$ \{ B \in \mathcal{B} \mid U \cap B \neq \varnothing \} $$
	de forma que este conjunto también debe ser finito. Así, para todo $x$ en $X$ existe un entorno que interseca solo a una cantidad finita de elementos de $\mathcal{B}$.
\end{proof}

\begin{mybox}
	\textbf{8. } Sean $f: A \rightarrow B$, $g: B \rightarrow D$ funciones continuas. La función $f \times g : A \times C \rightarrow B \times D$ definida mediante la ecuación
	$$ (f \times g)(a, c) = (f(a), g(c)) $$
	es continua. 
\end{mybox}	
\begin{proof}
	Sea $U \times V$ un elemento básico de la topología sobre $B \times D$. Entonces, 
	$$ (f \times g)^{-1}(U \times V) = f^{-1}(U) \times g^{-1}(V) .$$
	Puesto que $f$ y $g$ son continuas, se sigue que los conjuntos  $f^{-1}(U)$ y $ g^{-1}(V)$ son abiertos en $A$ y $C$, respectivamente, de forma que su producto es abierto en $A \times C$.  Así, la preimagen de todo elemento básico es abierta, de lo que se sigue que $f \times g$ es continua. 
\end{proof}

\newpage

\begin{mybox}
	\textbf{9. } Consideremos una función $F: X \times Y \rightarrow Z$. Decimos que $F$ es \textbf{continua en cada variable separadamente} si \\
	
	(i) para cada $y_{0} \in Y$, la función $h_{y_{0}}: X \rightarrow Z$ definida mediante $h(x) = F(x, y_{0})$ es continua \\
	
	(ii) y para cada $x_{0} \in X$, la función $k_{x_{0}}: Y \rightarrow Z$ definida mediante $k(y) = F(x_{0}, y)$ es continua. \\
	
	Si $F$ es continua, se sigue que $F$ es continua en cada variable separadamente. 
\end{mybox}	
\begin{proof}
	$\bullet$ Sea $y_{0} \in Y$ arbitrario y supongamos que $F$ es continua. Si $W$ es abierto en $Z$, queremos mostrar que $h_{y_{0}}^{-1}(W)$ es abierto en $X$. Por definición,
	
	$$ h^{-1}(W) = \{ x \in X \mid h(x) =  F(x, y_{0}) \in W \} $$
	
	$$ F^{-1}(W) = \{ (x, y) \in X \times Y \mid F(x, y) \in W \} .$$
	
	Por la continuidad de $F$, este último conjunto es abierto. Notamos que 
	
	$$ h^{-1}(W) \times \{ y_{0} \} = F^{-1}(W) \cap (X \times \{ y_{0} \}) .$$
	
	El conjunto anterior es abierto en el subespacio $X \times \{ y_{0} \}$ de $X \times Y$. Ahora consideremos la función $f:X \rightarrow X \times \{ y_{0} \}$ dada por 
	$$ f(x) = (x, y_{0}). $$
	Esta función es continua, dado que sus funciones coordenadas son continuas: la primera es la función identidad de $X$ a $X$ y la segunda es la función constante de $X$ a $\{ y_{0} \}$. Ahora, puesto que $h^{-1}(W) \times \{ y_{0} \}$ es abierto en $X \times \{ y_{0} \}$ y $f$ es continua, se sigue que 
	$$ f^{-1}( h^{-1}(W) \times \{ y_{0} \} ) = h^{-1}(W)$$
	es abierto en $X$.  \\
	
$\bullet$ Un argumento análogo muestra la continuidad de $F$ implica que para cada $x_{0}$ en $X$, la función $k$ es continua. 
\end{proof}

\newpage

\begin{mybox}
	\textbf{10. } Sea $A \subseteq X$ y $f: A \rightarrow Y$ continua, donde $Y$ es un espacio de Hausdorff. Si $f$ se puede extender a una función continua $g: \bar{A} \rightarrow Y$, entonces $g$ es única. 
\end{mybox}	

\begin{proof}
Supongamos que existen dos funciones continuas $g_{1}, g_{2}: \bar{A} \rightarrow Y$ tales que
	$$ \restr{g_{1}}{A} = \restr{g_{2}}{A} = f.  $$
	Puesto que $g_{1}$ y $g_{2}$ son continuas y $Y$ es Hausdorff, se sigue que
	$$ C = \{ x \in \bar{A} \mid g_{1}(x) = g_{2}(x) \} $$
	es cerrado en $\bar{A}$, de lo que se sigue que es cerrado en $X$. Notamos que $A \subseteq C$, de modo que se debe cumplir que $\bar{A} \subseteq C$. Así, $\bar{A} = C$ y se sigue que $g_{1} = g_{2}$.
\end{proof}

\begin{proof}
	También podemos proceder por contradicción. Supongamos que existe un $x \in \bar{A}$ tal que $g_{1}(x) \neq g_{2}(x)$. Usando el hecho que $Y$ es de Hausdorff, deducimos que existen entornos $U$ de $g_{1}(x)$ y $V$ de $g_{2}(x)$ tales que $U \cap V = \varnothing$. Por la continuidad de $g_{1}$ y de $g_{2}$, se sigue que el conjunto
	$$ g_{1}^{-1}(U) \cap g_{2}^{-1}(V) $$
	es un entorno de $x$. Puesto que $x \in \bar{A}$, se sigue que existe un $x'$ en
	$$ A \cap g_{1}^{-1}(U) \cap g_{2}^{-1}(V) $$
	A partir de esto podemos deducir que $U$ y $V$ no son disjuntos: contienen el punto $g_{1}(x') = g_{2}(x')$, lo cual contradice nuestra suposición anterior. Entonces, debe  ser el caso que $g_{1} = g_{2}$. 
\end{proof}
	
\begin{mybox}
	\textbf{11.} Si $f$ y $g$ son dos funciones continuas que van de $X$ a un espacio de Hausdorff $Y$, el conjunto
	$$ W =  \{ x \in X \mid f(x) = g(x)  \} $$
	es cerrado en $X$.
\end{mybox}
\begin{proof}
	Mostramos que 
	$$ X \backslash W = \{ x \in X \mid f(x) \neq g(x) \} $$
	es abierto. Si $x$ está en $X \backslash W$, se sigue que $f(x) \neq g(x)$. Puesto que $Y$ es Hausdorff, existen entornos disjuntos $U$ de $f(x)$ y $V$ de $g(x)$. Definamos 
	$$ O := f^{-1}(U) \cap g^{-1}(V). $$
	Entonces, $x \in O$.  Notamos que para cualquier $y$ en $O$, se debe cumplir que $f(y) \neq g(y)$: esto se deduce del hecho que $U$ y $V$ son disjuntos. Así, concluimos que $O \subseteq X \backslash W$. \\
	Para un punto arbitrario de $X \backslash W$, hemos encontrado un entorno contenido en este conjunto, de lo que se sigue que es abierto. 	
\end{proof}	

\textbf{Una demostración fallida. } Al ver el problema anterior, una idea que puede surgir es usar la diagonal $\Delta$ de $Y$. Definamos $h: X \rightarrow Y \times Y$ como 
$$ h(x) = ( f(x), g(x) ). $$
Puesto que ambas funciones coordenadas son continuas, se sigue que $h$ es continua. Del hecho que $Y$ es Hausdorff, podemos deducir que la digonal $\Delta$ es cerrada en $Y$. Si se cumple que
$$ h^{-1}(\Delta) = W $$
habremos demostrado que $W$ es cerrado. La demostración que $W \subseteq h^{-1}(\Delta)$ no presenta problemas: todos los puntos de $W$ son enviados por $h$ a la diagonal $\Delta$. 
Sin embargo, cuando queremos demostrar que $h^{-1}(\Delta) \subseteq W$, nos encontramos con el problema que $f$ y $g$ pueden no ser sobreyectivas: puede existir $y \in Y$ al que no llega ni $f$ ni $g$, de forma que $(y, y)$ estaría en $\Delta$ pero no en $h(W)$.\\

La proposición siguiente es una generalización $\mathbf{11}$ en §$\mathbf{1}$.

\begin{mybox}
	\textbf{12. } Supongamos que $f$ y $g$ son funciones continuas que van de un espacio $X$ a un espacio de Hausdorff $Y$. Si $A$ es un subconjunto denso de $X$ y $f$ y $g$ son iguales en todos los puntos de $A$, se sigue que son iguales en todos los puntos de $X$. 
\end{mybox}	
\begin{proof}
	Por el problema anterior, podemos deducir que 
	$$ W = \{ x \in X \mid f(x) = g(x) \}$$
	es cerrado en $X$. Notemos que $A \subseteq W$, de modo que $X = \text{cl}(A) \subseteq W$. Se sigue que $f = g$. 
\end{proof}

Lo anterior se puede reformular como que una función continua que toma valores en un espacio de Hausdorff está determinada por sus valores en un conjunto denso. \\

\begin{mybox}
	\textbf{13. } Sea $\{A_{i} \}_{i \in I}$ es una colección localmente finita de conjuntos cerrados en un espacio $X$. Entonces, $\bigcup_{i \in I} A_{i}$ es un conjunto cerrado. 
\end{mybox}	

\begin{proof}
	Mostramos que $X \backslash \bigcup_{i \in I} A_{i}$ es un conjunto abierto. Sea $x$ un elemento de $ X \backslash \bigcup_{i \in I} A_{i}$ y $U$ el entorno de $x$ que interseca solo a una cantidad finita de elementos de $\{A_{i} \}_{i \in I}$. \\
	
	\textit{Caso I. } $U$ no interseca a ningún elemento de $\{A_{i} \}_{i \in I}$. Entonces, $U \subseteq X \backslash \bigcup_{i \in I} A_{i}$. \\
	
	\textit{Caso II. } Supongamos que $U$ interseca solo a los conjuntos $A_{1}, \ldots, A_{n}$. Notamos que $\bigcup_{i=1}^{n} A_{i}$, al ser una unión finita de conjuntos cerrados, es un conjunto cerrado. Entonces,
	$$U \backslash \bigcup_{i = 1}^{n} A_{i} $$
	es un conjunto abierto que cumple que
	$$ x \in U \backslash \bigcup_{i = 1}^{n} A_{i} \subseteq  X \backslash \bigcup_{i \in I} A_{i}$$
	de lo que concluimos que $X \backslash \bigcup_{i \in I} A_{i}$ es abierto. 
\end{proof}

\begin{mybox}
	\textbf{14. }Sea $X$ un espacio topológico y $A \subseteq X$. Si $U$ es un abierto tal que $U \cap A = \varnothing$, entonces $U \cap \bar{A} = \varnothing$.
\end{mybox}	
\begin{proof}
	Notamos que el conjunto $X \backslash U$ es cerrado y que $A \subseteq X \backslash U$, de lo que se sigue que $\bar{A} \subseteq X \backslash U$ y que $ \bar{A} \cap U = \varnothing$. 
\end{proof}

\end{document}