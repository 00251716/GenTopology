\documentclass{article}
\usepackage[utf8]{inputenc}
\usepackage[spanish]{babel}
\usepackage{amsthm}
\usepackage{amssymb}
\usepackage{amsmath}
\usepackage{fancyhdr}
\usepackage{graphicx}
\usepackage[dvipsnames, table]{xcolor}
\usepackage[framemethod=tikz]{mdframed}
\usepackage{multicol}
\usepackage{tabularx}
\usepackage{pifont}
\setlength{\tabcolsep}{3pt}
\decimalpoint
\newcommand{\xmark}{\ding{55}}
\definecolor{mycolor}{rgb}{0.122, 0.435, 0.698}

\newmdenv[innerlinewidth=0.5pt, roundcorner=4pt,linecolor=mycolor,innerleftmargin=6pt,
innerrightmargin=6pt,innertopmargin=6pt,innerbottommargin=6pt]{mybox}


\setlength\parindent{0pt}

\newtheorem{definition}{Definición}
\newtheorem{proposition}{Proposición}

\renewcommand{\headrulewidth}{0.4pt}

\begin{document}
\date{27 de marzo - 10 de abril de 2020}
\title{ \textbf{Topología} \\
Semanas 4 y 5: Continuidad y homeomorfismos}
\maketitle	

\section*{§1}

\begin{mybox}
	\textbf{2. } Sean $X$ y $Y$ espacios topológicos y supongamos que $\mathcal{B}$ es una base para la topología sobre $Y$. Una función $f: X \rightarrow Y$ es continua si y solo si $f^{-1}(B)$ es abierto para todo $B \in \mathcal{B}$.
\end{mybox}	
\begin{proof}
	$(\Rightarrow)$ Si $f$ es continua, se sigue que la preimagen de cualquier abierto en $Y$ es abierta en $X$. En particular, la preimagen de cualquier elemento de la base $\mathcal{B}$ es abierta.\\
	$(\Leftarrow)$ Ahora tomemos un abierto $V$ en $Y$ y mostremos que $f^{-1}(V)$ es abierto en $X$. Entonces,
	$$ V = \bigcup_{i \in J} B_{i} ,$$
	de forma que 
	$$ f^{-1}(V) = \bigcup_{i \in J} f^{-1}(B_{i}) .$$
	Al ser la unión de conjuntos abiertos, se sigue que $f^{-1}(V)$ es abierto. 
\end{proof}

\begin{mybox}
	\textbf{11. } Sea $f: \mathbb{R} \rightarrow \mathbb{R}$ una función continua tal que $f(q) = 0$ para todo racional $q$. Mostrar que $f(x) = 0$ para todo $x \in \mathbb{R}$.  	
\end{mybox}	

\begin{proof}
	Puesto que $f$ es continua, se cumple que 
	$$ f(\text{cl}(\mathbb{Q})) \subseteq \text{cl}(f(\mathbb{Q})), $$
	es decir,
	$$ f(\mathbb{R}) \subseteq \{ 0 \}. $$
	Por la hipótesis, sabemos que $\{0 \} \subseteq f(\mathbb{R})$. A partir de esto, deducimos que $f(\mathbb{R}) = \{ 0 \}$.
\end{proof}

\begin{proof}
	Sea $x$ irracional y supongamos que $f(x) \neq 0$. Puesto que $\mathbb{R}$ es $T_{1}$, podemos encontrar un entorno $U$ de $f(x)$ que no contenga a $0$. Se sigue, por la continuidad de $f$, que $f^{-1}(U)$ es abierto. Del hecho que $U$ es abierto, podemos deducir que existe un racional $q$ en $f^{-1}(U)$. Esto implicaría que $0 = f(q) \in U$, lo cual es una contradicción. 
\end{proof}

\section*{§2 }

\begin{mybox}
	\textbf{2. }  Supongamos que $f: X \rightarrow Y$ es continua. Si $x$ es un punto límite del subconjunto $A$ de $X$, ¿es necesariamente cierto que $f(x)$ es un punto límite de $f(A)$? 
\end{mybox}	

$\bullet$ No es necesariamente cierto. Podría ser que el conjunto $Y$ no tenga puntos límite. Por ejemplo, $f: \mathbb{R} \rightarrow \mathbb{R}$ dada por $f(x) = 2$ es una función continua. Entonces, todo real $r$ es un punto límite de $\mathbb{R}$, pero $f(r) = 2$ no es un punto límite de ${2}$, porque este conjunto no tiene puntos límite.  

\begin{mybox}
	\textbf{3. } Consideremos $\text{id}: (X, \mathcal{T}_{1}) \rightarrow (X, \mathcal{T}_{2})$. Se cumplen las siguientes proposiciones: \\
	
	(i) La función $\text{id}$ es continua si y solo si $\mathcal{T}_{1}$ es más fina que $\mathcal{T}_{2}$. \\
	
	(ii) La función id es un homeomorfismo si y solo si $\mathcal{T}_{1} = \mathcal{T}_{2}$.  
\end{mybox}	
\begin{proof}
	(i) $(\Rightarrow)$ Supongamos que id es continua y consideremos un abierto $V$ en $\mathcal{T}_{2}$. Entonces, $\text{id}^{-1}(V) = V$ es abierto en $\mathcal{T}_{1}$, de modo que $\mathcal{T}_{1}$ debe ser más fina que $\mathcal{T}_{2}$. \\
	$(\Leftarrow)$ Ahora supongamos que $\mathcal{T}_{1}$ es más fina que $\mathcal{T}_{2}$. Entonces, al tomar un abierto $V$ de $\mathcal{T}_{2}$, sabemos que $\text{id}^{-1}(V) = V$ también será abierto en $\mathcal{T}_{1}$, de lo que concluimos que id es continua. \\
	
	(ii) $(\Rightarrow)$ Supongamos que id es un homeomorfismo. Entonces, si $U \in \mathcal{T}_{1}$, sabemos que $\text{id}(U) = U \in \mathcal{T}_{2}$, de forma que $\mathcal{T}_{1} \subseteq \mathcal{T}_{2}$.   De forma análoga, si $V \in \mathcal{T}_{2}$, $\text{id}^{-1}(V) = V$ es abierto en $\mathcal{T}_{1}$, de manera que $\mathcal{T}_{2} \subseteq \mathcal{T}_{1}$. \\
	$(\Leftarrow)$ Si $\mathcal{T}_{1} = \mathcal{T}_{2}$, se sigue inmediatamente $\text{id} = \text{id}^{-1}$ es continua.  
\end{proof}

\newpage

\begin{mybox}
	\textbf{5. } Todos los subespacios $(a, b)$  de $\mathbb{R}$ son homeomorfos. Similarmente, todos los subespacios de $\mathbb{R}$ de la forma $[a, b]$ son homeomorfos.
\end{mybox}	
\begin{proof}
	$\bullet$ Sea $(a, b)$ arbitrario. Mostramos que $(0,1) \cong (a, b)$. Consideremos la función $f: (0, 1) \rightarrow (a, b)$ dada por $f(x) = a(1-x) + bx$. La función $g: (a, b) \rightarrow (0, 1)$ dada por $g(x) = (x- a)/ (b - a)$ es inversa de $f$, de lo que deducimos que $f$ es biyectiva. \\
	 Ahora mostramos que $f$ es continua. El conjunto 
	$$ \mathcal{B} = \{ (a, b) \cap (c, d) \subseteq \mathbb{R} \mid c < d \} = \{ (c, d) \subseteq \mathbb{R} \mid a \leq c < d \leq b \}$$
	es una base para la topología sobre $(a, b)$.  Consideremos $(c, d) \subseteq (a, b)$. Entonces, 
	$$ f^{-1}((c, d)) = \left( \frac{c - a}{b - a}, \frac{d -a}{b - a} \right) \subseteq (0, 1) $$
	es abierto en la topología sobre $(0, 1)$. Así, la preimagen de todo básico de $(a, b)$ es abierta, de lo que se sigue que $f$ es continua. \\
	Para mostrar que $g$ es continua, tomemos un intervalo $(k, l) \subseteq (0,1)$. Entonces,
	$$ g^{-1}((k, l )) = ( a(1- k) + bk, a(1- l) + bl ) \subseteq (a, b), $$
	de forma que $g^{-1}$ es continua. \\
	Así, hemos demostrado que $(0, 1) \cong (a, b)$. Puesto que la relación entre dos espacios de ser homeomorfos es una relación de equivalencia, podemos deducir que todos los intervalos abiertos de $\mathbb{R}$ son homeomorfos. 
\end{proof}

\begin{mybox}
	\textbf{10. } Sean $f: A \rightarrow B$, $g: B \rightarrow D$ funciones continuas. La función $f \times g : A \times C \rightarrow B \times D$ definida mediante la ecuación
	$$ (f \times g)(a, c) = (f(a), g(c)) $$
	es continua. 
\end{mybox}	
\begin{proof}
	Sea $U \times V$ un elemento básico de la topología sobre $B \times D$. Entonces, 
	$$ (f \times g)^{-1}(U \times V) = f^{-1}(U) \times g^{-1}(V) .$$
	Puesto que $f$ y $g$ son continuas, se sigue que los conjuntos  $f^{-1}(U)$ y $ g^{-1}(V)$ son abiertos en $A$ y $C$, respectivamente, de forma que su producto es abierto en $A \times C$.  Así, la preimagen de todo elemento básico es abierta, de lo que se sigue que $f \times g$ es continua. 
\end{proof}

\newpage

\begin{mybox}
	\textbf{11. } Consideremos una función $F: X \times Y \rightarrow Z$. Decimos que $F$ es \textbf{continua en cada variable separadamente} si \\
	
	(i) para cada $y_{0} \in Y$, la función $h_{y_{0}}: X \rightarrow Z$ definida mediante $h(x) = F(x, y_{0})$ es continua \\
	
	(ii) y para cada $x_{0} \in X$, la función $k_{x_{0}}: Y \rightarrow Z$ definida mediante $k(y) = F(x_{0}, y)$ es continua. \\
	
	Si $F$ es continua, se sigue que $F$ es continua en cada variable separadamente. 
\end{mybox}	
\begin{proof}
	$\bullet$ Sea $y_{0} \in Y$ arbitrario y supongamos que $F$ es continua. Si $W$ es abierto en $Z$, queremos mostrar que $h_{y_{0}}^{-1}(W)$ es abierto en $X$. Por definición,
	
	$$ h^{-1}(W) = \{ x \in X \mid h(x) =  F(x, y_{0}) \in W \} $$
	
	$$ F^{-1}(W) = \{ (x, y) \in X \times Y \mid F(x, y) \in W \} .$$
	
	Por la continuidad de $F$, este último conjunto es abierto. Notamos que 
	
	$$ h^{-1}(W) \times \{ y_{0} \} = F^{-1}(W) \cap (X \times \{ y_{0} \}) .$$
	
	El conjunto anterior es abierto en el subespacio $X \times \{ y_{0} \}$ de $X \times Y$. Ahora consideremos la función $f:X \rightarrow X \times \{ y_{0} \}$ dada por 
	$$ f(x) = (x, y_{0}). $$
	Esta función es continua, dado que sus funciones coordenadas son continuas: la primera es la función identidad de $X$ a $X$ y la segunda es la función constante de $X$ a $\{ y_{0} \}$. Ahora, puesto que $h^{-1}(W) \times \{ y_{0} \}$ es abierto en $X \times \{ y_{0} \}$ y $f$ es continua, se sigue que 
	$$ f^{-1}( h^{-1}(W) \times \{ y_{0} \} ) = h^{-1}(W)$$
	es abierto en $X$.  \\
	
$\bullet$ Un argumento análogo muestra la continuidad de $F$ implica que para cada $x_{0}$ en $X$, la función $k$ es continua. 
\end{proof}

\newpage

\section*{§3 }

\begin{mybox}
	\textbf{1.} Si $f$ y $g$ son dos funciones continuas que van de $X$ a un espacio de Hausdorff $Y$, el conjunto
	$$ W =  \{ x \in X \mid f(x) = g(x)  \} $$
	es cerrado en $X$.
\end{mybox}
\begin{proof}
	Mostramos que 
	$$ X \backslash W = \{ x \in X \mid f(x) \neq g(x) \} $$
	es abierto. Si $x$ está en $X \backslash W$, se sigue que $f(x) \neq g(x)$. Puesto que $Y$ es Hausdorff, existen entornos disjuntos $U$ de $f(x)$ y $V$ de $g(x)$. Definamos 
	$$ O := f^{-1}(U) \cap g^{-1}(V). $$
	Entonces, $x \in O$.  Notamos que para cualquier $y$ en $O$, se debe cumplir que $f(y) \neq g(y)$: esto se deduce del hecho que $U$ y $V$ son disjuntos. Así, concluimos que $O \subseteq X \backslash W$. \\
	Para un punto arbitrario de $X \backslash W$, hemos encontrado un entorno contenido en este conjunto, de lo que se sigue que es abierto. 	
\end{proof}	

\textbf{Una demostración fallida. } Al ver el problema anterior, una idea que puede surgir es usar la diagonal $\Delta$ de $Y$. Definamos $h: X \rightarrow Y \times Y$ como 
$$ h(x) = ( f(x), g(x) ). $$
Puesto que ambas funciones coordenadas son continuas, se sigue que $h$ es continua. Del hecho que $Y$ es Hausdorff, podemos deducir que la digonal $\Delta$ es cerrada en $Y$. Si se cumple que
$$ h^{-1}(\Delta) = W $$
habremos demostrado que $W$ es cerrado. La demostración que $W \subseteq h^{-1}(\Delta)$ no presenta problemas: todos los puntos de $W$ son enviados por $h$ a la diagonal $\Delta$. 
Sin embargo, cuando queremos demostrar que $h^{-1}(\Delta) \subseteq W$, nos encontramos con el problema que $f$ y $g$ pueden no ser sobreyectivas: puede existir $y \in Y$ al que no llega ni $f$ ni $g$, de forma que $(y, y)$ estaría en $\Delta$ pero no en $h(W)$.\\

La proposición siguiente es una generalización $\mathbf{11}$ en §$\mathbf{1}$.

\begin{mybox}
	\textbf{2. } Supongamos que $f$ y $g$ son funciones continuas que van de un espacio $X$ a un espacio de Hausdorff $Y$. Si $A$ es un subconjunto denso de $X$ y $f$ y $g$ son iguales en todos los puntos de $A$, se sigue que son iguales en todos los puntos de $X$. 
\end{mybox}	
\begin{proof}
	Por el problema anterior, podemos deducir que 
	$$ W = \{ x \in X \mid f(x) = g(x) \}$$
	es cerrado en $X$. Notemos que $A \subseteq W$, de modo que $X = \text{cl}(A) \subseteq W$. Se sigue que $f = g$. 
\end{proof}

Lo anterior se puede reformular como que una función continua que toma valores en un espacio de Hausdorff está determinada por sus valores en un conjunto denso. \\

Willard, General Topology, 48

\end{document}