\documentclass{article}
\usepackage[a4paper,bindingoffset=0.2in,%
left=1in,right=1in,top=1in,bottom=1in,%
footskip=.25in]{geometry}
\usepackage[utf8]{inputenc}
\usepackage[spanish]{babel}
\usepackage{amsthm}
\usepackage{amssymb}
\usepackage{amsmath}
\usepackage{fancyhdr}
\usepackage{graphicx}
\usepackage[dvipsnames, table]{xcolor}
\usepackage[framemethod=tikz]{mdframed}
\usepackage{multicol}
\usepackage{tabularx}
\usepackage{pifont}
\setlength{\tabcolsep}{3pt}
\decimalpoint
\newcommand{\xmark}{\ding{55}}
\definecolor{mycolor}{rgb}{0.122, 0.435, 0.698}

\newmdenv[innerlinewidth=0.5pt, roundcorner=4pt,linecolor=mycolor,innerleftmargin=6pt,
innerrightmargin=6pt,innertopmargin=6pt,innerbottommargin=6pt]{mybox}

\newcommand\restr[2]{{% we make the whole thing an ordinary symbol
		\left.\kern-\nulldelimiterspace % automatically resize the bar with \right
		#1 % the function
		\vphantom{\big|} % pretend it's a little taller at normal size
		\right|_{#2} % this is the delimiter
}}

\setlength\parindent{0pt}

\newtheorem{definition}{Definición}
\newtheorem{proposition}{Proposición}

\renewcommand{\headrulewidth}{0.4pt}

\begin{document}
\date{10 - 24 de abril de 2020}
\title{ \textbf{Topología} \\
Semanas 6 y 7: Productos cartesianos arbitrarios}
\maketitle	

\begin{definition}
	Sea $J$ un conjunto de índices. Una $J$-tupla de elementos de un conjunto $X$ es una función 
	$$ \mathbf{x}: J \rightarrow X. $$
	Si $\alpha$ es un elemento de $X$, a menudo denotamos el valor de $\mathbf{x}$ en $\alpha$ como $x_{\alpha}$ en lugar de $\mathbf{x}(\alpha)$ y lo denominamos la $\alpha$-ésima coordenada de $\mathbf{x}$. A menudo denotamos $\mathbf{x}$ como 
	$$ (x_{\alpha})_{\alpha \in J} .$$
	Además, denotamos el conjunto de todas las $J$-tuplas de elementos de $X$ como $X^{J}$. De manera más formal,
	$$ X^{J} = \{ \mathbf{x} \subseteq J \times X \mid \mathbf{x}: J \rightarrow X \}. $$ 
\end{definition}

$\bullet \textbf{Ejemplo 1. }$ Si $X = \mathbb{R}$ y $J = 3$, $\mathbb{R}^{3}$ es el conjunto de todas las $3$-tuplas de números reales. \\

$\bullet \textbf{Ejemplo 2.}$ Si $X = \mathbb{C}$ y $J = \mathbb{Z}$. Un elemento de $\mathbb{C}^{\mathbb{Z}}$ podría ser, por ejemplo, la función $\mathbf{x}$ dada por 
$$ x_{k} = e^{-ik} $$
para cualquier $k \in \mathbb{Z}$, que también podemos denotar como
$$ (\ldots,e^{2i}, e^{i}, 1, e^{-i}, e^{-2i} ,\ldots) $$
o como $(e^{-ik})_{k \in \mathbb{Z}}$. En general, los elementos de $\mathbb{C}^{\mathbb{Z}}$ son las $\mathbb{Z}$-tuplas de números complejos. \\

$\bullet \textbf{Ejemplo 3. }$ Si $X = J = \mathbb{R}$, obtenemos
$$ \mathbb{R}^{\mathbb{R}} = \{ f \subseteq \mathbb{R} \times \mathbb{R} \mid f: \mathbb{R} \rightarrow \mathbb{R} \} .$$

\begin{definition}
	Sea $\{ A_{i} \}_{i \in J}$ una famila indexada de conjuntos. Definimos el producto cartesiano $\prod_{i \in J} A_{i}$ como 
	$$ \prod_{i \in J} A_{j} := \{ \mathbf{x}: J \rightarrow \bigcup_{i \in J} A_{i}  \mid \forall i \in I \hspace{0.2cm} ( \mathbf{x}(i) \in A_{i} ) \}.$$
	Notamos que si $X = A_{i}$ para toda $i \in J$, el producto cartesiano coincide con el conjunto $X^{J}$ de todas las $J$-tuplas de elementos de $X$. 
\end{definition}

\begin{definition}
	
\end{definition}

\end{document}